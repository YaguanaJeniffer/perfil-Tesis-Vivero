
En esta investigación, se utilizarán sensores de humedad del suelo, temperatura
y humedad ambiental debido a su capacidad para recopilar datos precisos y en
tiempo real sobre condiciones específicas. Por lo tanto, permite ajustes y
decisiones basadas en datos concretos en diversos entornos y aplicaciones. Es
por ello que, el uso de un microcontrolador es ventajoso gracias a su
versatilidad y capacidad para procesar datos provenientes de múltiples fuentes
como los sensores.

\bigbreak
La lógica difusa de tipo mandami es necesaria por su capacidad para modelar comportamientos en entornos complejos y variables. De esta manera, es aplicable en escenarios donde la adaptación precisa a diferentes condiciones es esencial.

\bigbreak
La elección de una base de datos relacional se debe a su capacidad para organizar y relacionar datos de manera estructurada. Por efecto, esta estructura asegura la integridad y coherencia de la información, permitiendo consultas y análisis detallados en diferentes contextos.

\bigbreak
Las aplicaciones móviles se ah seleccionado porque ofrecen acceso remoto a datos y funcionalidades, permitiendo a los usuarios interactuar desde cualquier lugar. Por consiguiente, estas aplicaciones facilitan la movilidad y la conveniencia en diversos contextos.

\bigbreak
El uso de una API basada en arquitectura REST es valioso por su capacidad para facilitar la interacción entre sistemas. Dicho de otro modo, esta arquitectura permite la comunicación eficiente y estándar entre diferentes plataformas y servicios.

\bigbreak
Se opta por la arquitectura MVC porque proporciona una estructura clara y organizada para el desarrollo de aplicaciones.Es por ello que, esta separación de la lógica de negocio de la lógica de presentación simplifica la gestión de procesos permiten un desarrollo modular y mantenible en una amplia gama de aplicaciones y sistemas.

\bigbreak
Finalmente, la metodología XP (Extreme Programming) se elige por su capacidad para adaptarse ágilmente a los cambios que puedan surgir durante el desarrollo de un proyecto. Cabe resaltar que se enfoca en la flexibilidad, la retroalimentación continua y la adaptación a requisitos cambiantes.

