\subsection{Recolección de información}
La Recolección de Información se llevará a cabo a través de entrevistas con los
responsables directos del Vivero Michita. Estas entrevistas estarán enfocadas
en obtener una comprensión detallada de los métodos actuales de riego, las
prácticas de gestión de agua, los desafíos enfrentados y las necesidades
específicas del vivero.

\bigbreak
Se utilizarán preguntas estructuradas para capturar
información relevante sobre el consumo de agua por parte de las plantas, las
condiciones ambientales y los procedimientos de cultivo. Las entrevistas se
llevarán a cabo de manera individual con cada responsable, permitiendo una
exploración exhaustiva de los distintos aspectos de la gestión del agua en el
vivero.

\bigbreak
Ademas, se utilizara un protocolo técnico diseñado para la adquisición de datos en el Vivero Michita, en el contexto del desarrollo de este prototipo, se ejecutará mediante una secuencia de pasos.

\subsubsection*{Sensor de Humedad del Suelo}
Se ubicará a una profundidad de 12 centímetros en áreas representativas de cada categoría de plantas en el vivero Michita. Se dispondrá un sensor por cada categoría de plantas para capturar las variaciones específicas de humedad en sus respectivas áreas.

\subsubsection*{Sensor de Temperatura y Humedad Ambiental}
Se instalarán a diferentes alturas en distintos sectores del vivero para registrar las variaciones de temperatura y humedad a lo largo del día. Se colocarán de manera estratégica para abarcar áreas representativas de las distintas categorías de plantas.

\subsubsection*{Consideraciones Especiales}
Se ajustará la disposición de los sensores conforme a las variaciones topográficas y ambientales presentes en el vivero Michita. Se realizarán mediciones piloto para validar la eficacia y precisión de la ubicación de los sensores antes de su despliegue definitivo.