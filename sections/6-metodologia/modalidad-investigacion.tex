\subsection{Modalidad de investigación}
En esta sección, se proporcionará una descripción detallada de las modalidades de investigación adoptadas para el proyecto. A continuación se detalla cada una.

\bigbreak
\textbf{Investigación cuantitativa}\\
Este proyecto se apoya en una modalidad de investigación cuantitativa al emplear variables numéricas para llevar a cabo procesos estadísticos y presentar datos. Su enfoque se centra en la medición precisa de variables como niveles de humedad, temperaturas y patrones de riego, con el propósito de establecer relaciones cuantificables que mejoren la gestión del agua.

\bigbreak
\textbf{Investigación experimental}\\
Se plantea una investigación de tipo experimental para este proyecto, ya que se busca evaluar la efectividad y la precisión del sistema de gestión de riego. Esto implica la realización de pruebas controladas en viveros, variando los niveles de riego de acuerdo con los datos obtenidos por la lógica difusa.

\bigbreak
\textbf{Investigación aplicada}\\
Se trata de una investigación aplicada, ya que se utilizarán los conocimientos adquiridos a lo largo de la carrera para el desarrollo y ejecución de la propuesta. Esto implica la aplicación práctica de teorías, métodos y herramientas aprendidas en un entorno real para la creación del sistema de gestión de riego en el prototipo de vivero.
