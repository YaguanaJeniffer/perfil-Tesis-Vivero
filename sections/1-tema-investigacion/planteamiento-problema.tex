\subsection{Planteamiento del problema}
El agua desempeña un papel fundamental en la agricultura, ya que es uno de los
principales consumidores de este recurso hídrico. Según la Organización de las
Naciones Unidas para la Alimentación y la Agricultura (FAO), la actividad
agrícola emplea aproximadamente el 70\% del suministro mundial de agua dulce
\cite{noauthor_gestion_2018}. Es por ello, que la cantidad de agua utilizada en
los campos de cultivo varía según diversos factores, como el tipo de cultivo,
las condiciones climáticas, la calidad del suelo y el método de riego empleado.
Por lo tanto, la FAO menciona que para aumentar la eficiencia en el uso del
agua, es necesario reducir su uso entre el 25\% y el 40\% de este recurso
\cite{noauthor_gestion_2018}.

\bigbreak
En Ecuador las condiciones climáticas varían significativamente entre regiones y estaciones \cite{temperatura_nodate}.
En este sentido, la variabilidad afecta directamente la cantidad de agua disponible, algo fundamental para el óptimo desarrollo de los cultivos vegetales en viveros,
incluyendo plantas ornamentales, forestales y frutales. Es por eso, que en el artículo \cite{c_estudio_2018},
se menciona que en Mulaló el 83\% de los agricultores están haciendo un uso excesivo del agua de riego, debido a que disponen de cantidades superiores a las necesarias tanto para los cultivos como para la extensión de tierra irrigada.

\bigbreak
Por tanto, la gestión ineficiente del agua puede disminuir la productividad agrícola en viveros.
De este modo, cuando los cultivos no reciben la cantidad necesaria de agua, se reduce la eficiencia en la producción
y se compromete el desarrollo deficiente de las plantas ornamentales, forestales y frutales.
Es por ello, que estos desafíos resaltan la importancia de mejorar la eficiencia en la administración del agua
en este ámbito específico.

\bigbreak
El problema que se plantea y que se busca solucionar a través de esta investigación es el desperdicio de agua
que en ciertos momentos se pudiera dar. Si bien es cierto, este problema se da por la falta de un sistema de monitoreo
y control del consumo de agua.

\bigbreak
Por lo tanto, se menciona las causas de esta situación:
\begin{enumerate}
      \item Riego manual: Todos los días se procede a regar las plantas del vivero de
            manera manual, lo que hace imposible controlar de manera precisa la cantidad de
            agua que se consume.
      \item Escasez de Información: El vivero no dispone de información detallada sobre la
            cantidad de agua que requiere cada especie de planta.
      \item Ausencia de un Sistema: El vivero no cuenta con un sistema que le permita la
            toma de decisiones frente al uso excesivo de agua.
\end{enumerate}
También es necesario hablar de las consecuencias que generan estas situaciones,
entre éstas mencionamos las siguientes:
\begin{enumerate}
      \item Desperdicio de agua: El riego excesivo puede generar un desperdicio de agua, lo
            que puede tener un impacto negativo al medio ambiente.
      \item Aumento de los costos: El desperdicio de agua puede generar un aumento de los
            costos de producción del vivero.
      \item Reducción de la productividad: El riego excesivo puede reducir la productividad
            del vivero, ya que las plantas pueden sufrir estrés hídrico.
\end{enumerate}
Después de lo expuesto, se puede deducir de manera clara la importancia y el impacto positivo que
generará si se lleva a cabo el estudio de este problema.