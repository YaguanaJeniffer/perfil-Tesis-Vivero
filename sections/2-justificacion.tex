El manejo eficiente del agua en los viveros es importante, ya que al implementar un sistema de monitoreo y control del consumo de agua en el Vivero Michita no solo reducirá el desperdicio, sino que también impactará positivamente en el medio ambiente, la productividad y los costos de producción.

\bigbreak
De esta manera, se respalda esta necesidad a partir del impacto positivo que los agricultores experimentaron al adoptar sistemas avanzados de riego en Ecuador \cite{riego_tecnificado}. Visto de esta forma demuestra el potencial transformador que un Prototipo de Sistema de Monitoreo y Control basado en Lógica Difusa podría tener en el Vivero Michita.

\bigbreak
Es por ello que, la propuesta se destaca por su enfoque innovador que es el uso de la lógica difusa para el desarrollo del sistema de monitoreo y control del consumo de agua. Esta técnica de inteligencia artificial permite tomar decisiones a partir de valores imprecisos, adaptándose a las condiciones cambiantes del entorno, como las condiciones climáticas y las necesidades de las plantas.

\bigbreak
Los beneficios que se obtendrían con la implementación son notorios como:

\begin{enumerate}
    \item Contribuir a la conservación del agua: El desarrollo de un sistema de monitoreo y control del consumo de agua puede ayudar a reducir el desperdicio de agua, lo que contribuirá a la conservación de este recurso natural.
    \item Mejorar la productividad del vivero: El desarrollo de un sistema de monitoreo y control del consumo de agua puede ayudar a mejorar la productividad del vivero, ya que las plantas pueden recibir la cantidad de agua adecuada.
    \item Reducir los costos de producción: El desarrollo de un sistema de monitoreo y control del consumo de agua puede ayudar a reducir los costos de producción del vivero, ya que se reducirá el desperdicio de agua.
\end{enumerate}
