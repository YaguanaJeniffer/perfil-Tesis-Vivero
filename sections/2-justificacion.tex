El manejo eficiente del agua en los viveros es importante, dado que se establece como un componente esencial en la administración completa de estos terrenos destinados a la horticultura y al cultivo de plantas. Es por ello que al implementar un sistema que permita el monitoreo y control del consumo hídrico en el Vivero Michita, se logrará no solo reducir el desperdicio de agua, sino también generar un impacto positivo en varios aspectos. Estos aspectos incluyen al medio ambiente, la mejora en la productividad y la reducción de los costos de producción.

\bigbreak 
En \cite{riego_tecnificado}, los  agricultores ecuatorianos experimentaron un impacto positivo al optar por sistemas avanzados de riego. En este sentido se respalda la necesidad de implementar un sistema de monitoreo y control del consumo de agua. Es por eso que no solo busca mejorar la gestión del agua, sino también aumentar la eficiencia en el uso de este recurso vital.

\bigbreak 
Además, la implementación de este sistema en el Vivero Michita posibilitaría un control más efectivo de los recursos hídricos, permitiendo un riego más preciso y acorde a las necesidades de cada especie. Por consiguiente, su impacto económico también resultaría relevante al disminuir los costos asociados al uso ineficiente del agua y al mejorar la productividad de las plantas.

% \bigbreak
% Los beneficios que se obtendrían con la implementación son notorios como:

% \begin{enumerate}
%     \item Contribuir a la conservación del agua: El desarrollo de un sistema de monitoreo
%           y control del consumo de agua puede ayudar a reducir el desperdicio de agua, lo
%           que contribuirá a la conservación de este recurso natural.
%     \item Mejorar la productividad del vivero: El desarrollo de un sistema de monitoreo y
%           control del consumo de agua puede ayudar a mejorar la productividad del vivero,
%           ya que las plantas pueden recibir la cantidad de agua adecuada.
%     \item Reducir los costos de producción: El desarrollo de un sistema de monitoreo y
%           control del consumo de agua puede ayudar a reducir los costos de producción del
%           vivero, ya que se reducirá el desperdicio de agua.
% \end{enumerate}
