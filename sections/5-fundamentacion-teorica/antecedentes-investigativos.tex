\subsection{Antecedentes investigativos}
Luego de un análisis en repositorios digitales, plataformas académicas y
diversas bases de datos científicas, se ha recopilado una extensa variedad de
fuentes pertinentes. Estas fuentes han enriquecido y proporcionado un contexto
significativo al proyecto en consideración.

\bigbreak
En \cite{lema_holguin_implementacion_2018} se implemento un sistema automatizado de riego por goteo. Se utilizó una configuración de sensores para monitorear la humedad del sustrato, la conductividad eléctrica del suelo y el pH del agua, ubicados en contenedores de cultivo individuales con 24 plantas de tomate en cada uno. El proyecto presentó una solución integral para controlar y optimizar las condiciones de cultivo. Se implementó un invernadero que proporcionaba microclimas controlados para mitigar los impactos climáticos externos y mantener un entorno óptimo para el desarrollo de las plantas. Los sensores de temperatura, humedad, pH y conductividad eléctrica del sustrato se conectaron a dos controladores difusos tipo Mamdani, encargados de regular el riego y la administración de nutrientes en base a las lecturas obtenidas. En cuanto al hardware utilizado, se combinaron varios componentes específicos. Entre ellos se encuentran sensores como el Sensor Con-BTA y YL-68 para mediciones de pH y humedad, el Sensor Digital de Temperatura DS18B20, junto con tarjetas Arduino Mega y Leonardo con Ethernet Shield para control y comunicación de datos. Se integraron bombas de agua, electroválvulas y otros sensores, cada uno asignado a funciones específicas dentro del sistema de riego por goteo automatizado. El diseño del sistema incluyó la definición de reglas difusas para los controladores, los cuales ejecutaban acciones en función de las lecturas de los sensores. La comunicación se llevó a cabo mediante protocolos Modbus y Ethernet, y los datos se visualizaron a través de una interfaz hombre-máquina para monitorear y operar el sistema. Se realizaron mediciones periódicas mediante sensores y dispositivos de monitoreo para mantener las condiciones óptimas de cultivo. Los tiempos de respuesta de los controladores se mantuvieron dentro de los 5 segundos, comparados con tiempos obtenidos en Matlab. La gestión y visualización de datos se efectuó a través de una interfaz desarrollada en PHP integrada con la Ethernet Shield de Arduino. Esta combinación facilitó la comunicación de datos y la gestión de la base de datos en phpMyAdmin, permitiendo un análisis de las variables del proceso.

\bigbreak
En \cite{alcivar_dominguez_sistema_2018},se implementó un sistema automatizado de
riego basado en sensorización en los cultivos de ciclo corto. Este se compone de una variedad de componentes electrónicos clave. Entre ellos se encuentran sensores como el Módulo HL 69, el Módulo YL-83 y el DHT22, diseñados para recopilar datos precisos sobre la humedad del suelo, humedad ambiental y temperatura. Además, se emplea un conjunto de actuadores, incluyendo el Modulo Relé de 2 canales y el Modulo RTC, los cuales permiten la ejecución de acciones automáticas en respuesta a los datos recopilados. La centralidad del Arduino Mega como núcleo de control en esta configuración resulta fundamental, coordinando la información de los sensores y tomando decisiones para el riego eficiente de los cultivos. Por otro lado, el software desarrollado para este sistema se basa en la plataforma de Visual Studio Community 2015. Esta elección se respalda en la versatilidad que ofrece, especialmente a través de Windows Form, facilitando el diseño y la presentación de consultas y reportes generados por el sistema. La integración de una base de datos MySQL-2012 le permitió la creación de archivos de tipo app config para manejar los parámetros de conexión al servidor, facilitando la lectura y almacenamiento de información para el sistema. Cabe destacar que este proyecto siguió una metodología de desarrollo específica. La aplicación de escritorio se llevó a cabo bajo la metodología Cascada, asegurando una secuencia metódica desde el análisis de necesidades hasta la implementación física y la validación del sistema en el terreno. Por otro lado, la implementación del hardware se realizó mediante la metodología Free Hardware, lo que permitió una adaptabilidad y flexibilidad óptimas en la selección y ensamblaje de componentes. Este enfoque integral y sistemático garantizó la efectividad y adaptabilidad del sistema automatizado de riego, que logró resultados significativos como la reducción del consumo de agua en un 37.27\%, un aumento del 7.53\% en la productividad de los cultivos y un eficiente control de humedad del suelo.

\bigbreak
En \cite{salazar_irrigation_2013}, se desarrollo un sistema de Riego a través de Agentes Inteligentes usando la Tecnología Arduino. Este sistema esta enfocado en medir la humedad del suelo a través de sensores. Este sistema busca la efectividad y precisión en el riego, así como la activación y desactivación del sistema mediante lógica difusa. La arquitectura propuesta se basa en agentes inteligentes que interactúan entre sí y con el entorno, utilizando una combinación de lógica difusa y el modelo BDI (Believe, Desire, Intention). El proceso inicia con la placa Arduino MEGA, seleccionada por su capacidad de conectividad y control, en comunicación con diversos sensores, especialmente el sensor de humedad del suelo, y actuadores que controlan el riego. Los sensores recopilan datos sobre la humedad del suelo, que son procesados por un Agente de Campo y luego enviados al Agente de Control. Este último, un agente deliberativo, utiliza lógica difusa para tomar decisiones sobre el riego, basándose en valores de humedad para activar o desactivar el sistema. La lógica difusa se implementa utilizando un sistema MISO (entrada múltiple, salida única), datos de la humedad del suelo y su traducción en variables lingüísticas difusas. Esta arquitectura multiagente se desarrolla sobre la plataforma JADE, siguiendo el estándar FIPA para sistemas multiagente, y consta de tres agentes principales: el Maestro, que supervisa la base de datos y activa el sistema; el de Campo, encargado de recopilar y normalizar los datos de los sensores; y el de Control, responsable de tomar decisiones sobre el riego. La interacción entre Arduino y la computadora se realiza mediante una conexión serial, y se ha desarrollado una interfaz de software que muestra detalles de los sensores y el estado del sistema de riego, con modos manual y automático. Las pruebas del sistema se llevaron a cabo en un proyecto de guayabas taiwanesas, demostrando que el sistema de riego basado en agentes inteligentes permitió un monitoreo en tiempo real de las necesidades de agua de cada cultivo, adaptándose a las condiciones específicas del suelo.

\bigbreak
En \cite{hasan_implementation_2018} el autor se centra en la gestión óptima de la humedad del suelo. Este sistema automatizado utiliza sensores especializados para monitorear constantemente el contenido de humedad en el suelo, enviando datos al sistema cada segundo. Con base a la información recolectada, el sistema emplea lógica difusa para calcular el nivel de control requerido. Esto activa un método de control que ajusta una válvula en la línea de suministro de agua, permitiendo el flujo desde un tanque elevado estático hacia el campo. El proceso de riego se adapta dinámicamente según los niveles de humedad: si la humedad es baja, se inicia el riego a máxima velocidad, aumentando gradualmente hasta alcanzar el nivel deseado. A medida que la humedad aumenta, el sistema ajusta la válvula para reducir el flujo de agua, deteniéndolo al alcanzar el nivel óptimo. Incluso en este estado, el sistema sigue monitoreando y ajustando la válvula hasta que la humedad regresa al nivel bajo, reiniciando así el proceso. Este modelo de riego automatizado, basado en un algoritmo difuso, clasifica la humedad en seis clases: 'Muy bajo, bajo, medio, preciso, superior y desbordado'. Utiliza microcontroladores, motores y bombas para su funcionamiento. Los resultados detallados de las pruebas comparativas entre el método de riego tradicional y este modelo propuesto demostraron una reducción del 12.3\% en el uso de agua. Esto equivale a un ahorro de 81 litros por cada 100 metros cuadrados de área de cultivo, lo que representa un avance significativo hacia el uso eficiente del agua en sistemas de riego.  Este enfoque, además de su relevancia práctica en sistemas de riego, utiliza lógica difusa para una distribución más precisa y adecuada del agua.

\bigbreak
En \cite{munir_intelligent_2019}, se ha desarrollado un sistema de riego inteligente que se apoya en la lógica difusa y la tecnología blockchain para gestionar eficientemente el riego de las plantas. Utilizando datos provenientes de sensores que monitorean variables como temperatura, humedad, intensidad de luz, nivel de humedad del suelo y estado de la bomba de agua, este sistema toma decisiones en tiempo real respecto al riego de las plantas. Este sistema se compone de hardware como Arduino-UNO R3, Bread Board, sensores DHT-11 de temperatura y humedad, sensor YL-69 de humedad del suelo, entre otros. Estos dispositivos recopilan información crucial sobre las condiciones ambientales y del suelo, vital para determinar las necesidades hídricas de las plantas. La conectividad inalámbrica se logra mediante un módulo Wi-Fi ESP 8266-01, facilitando la transmisión de datos entre sensores y el servidor central. El proceso comienza con la adquisición de datos en tiempo real provenientes de los sensores, los cuales se transmiten al servidor para su procesamiento. La aplicación móvil para Android permite a los usuarios monitorear y controlar remotamente el sistema de riego, configurar horarios de riego, verificar la humedad del suelo y recibir recomendaciones personalizadas para el cuidado de las plantas. La metodología empleada se basa en Fuzzy Logic y Blockchain. La lógica difusa facilita la toma de decisiones sobre las necesidades de riego, mientras que la tecnología blockchain asegura la conectividad segura de dispositivos en un entorno IoT, permitiendo el acceso exclusivo a dispositivos confiables. Los resultados experimentales revelan una alta precisión del sistema. La mayoría de los experimentos alcanzan un 100\% de efectividad, es decir que la precisión general fue del 95.83\%.

\bigbreak
En \cite{al-ali_iot-solar_2019}, se menciona un sistema de riego inteligente alimentado por energía solar para la agricultura, denominado "IoT-solar energy powered smart farm irrigation system". El sistema recopila múltiples tipos de datos, incluyendo lecturas de humedad y temperatura del suelo, provenientes de sistemas de riego y componentes electrónicos. Sus objetivos principales se centran en el monitoreo remoto, control de humedad, temperatura, energía solar y riego automático en entornos agrícolas. Para lograr esto, se desarrollaron algoritmos de control y lógica difusa, los cuales permiten medir la eficiencia del riego, la adaptabilidad a condiciones climáticas variables y el consumo de energía solar. El sistema se basa en un controlador de National Instruments, myRIO, con un procesador ARM de doble núcleo y una matriz FPGA, conectado a una variedad de sensores como humedad del suelo, humedad y temperatura, sensor de flujo, interruptores de flotador y actuadores como bombas de achique y de diafragma. Es alimentado por energía solar durante el día, gestionada por un controlador de carga para una batería que suministra energía durante la noche o en condiciones de poca luz solar. La unidad DAQ-Sen, con sensores y actuadores con acceso a Internet, permite a los agricultores controlar el sistema a través de una aplicación móvil, integrando el concepto de IoT. El proceso de control se desglosa en tres modos: control local, monitoreo y control móvil, y control basado en lógica difusa. El último modo implica tomar decisiones sobre el encendido o apagado de las bombas basándose en lecturas de sensores y un algoritmo de control difuso. El sistema de riego inteligente propuesto se diseñó con la premisa de ser accesible y controlable desde cualquier ubicación y en cualquier momento, ofreciendo al usuario la capacidad de asumir el control total del sistema cuando así lo desee. Para lograr esta funcionalidad, se implementó un servidor web remoto que se respalda con una base de datos en lenguaje de consulta estructurado (SQL). Esta base de datos se encarga de almacenar de manera organizada y estructurada las mediciones provenientes de los sensores utilizados en el sistema.  El software del sistema se desarrolla utilizando LabView, con múltiples bucles ejecutándose en paralelo para el manejo de datos crudos de sensores, lógica de control difuso, cálculo del caudal y almacenamiento de datos para análisis posterior. Ademas, se utilizo un controlador de sistema en un chip de placa única con conectividad WiFi y conexiones a una celda solar para leer sensores de campo y emitir señales de comando para operar las bombas de riego. Se llevaron a cabo pruebas para verificar el funcionamiento del sistema en diferentes condiciones de humedad y temperatura del suelo, confirmando su eficacia y capacidad de adaptación.

\bigbreak
En \cite{orozco_jaramillo_diseno_2019} se ah diseñado e implementado una red de sensores para monitorear los niveles de radiación solar en la ciudad de Loja, se destaca la utilización de sensores GUVA-S12SD, Raspberry Pi 3B y una aplicación móvil desarrollada en Android Studio para llevar a cabo la medición y visualización de estos niveles. El proceso de adquisición de datos se realiza a través de sensores especializados, como el sensor GUVA-S12SD, dedicados a monitorear los niveles de radiación solar. Estos sensores capturan la información y la envían a una unidad central, la Raspberry Pi, que actúa como concentrador de datos. Posteriormente, la Raspberry Pi recibe, procesa y almacena los datos provenientes de los sensores en una base de datos MySQL. Aquí se conservan los valores recopilados a lo largo del tiempo, permitiendo un almacenamiento histórico de los niveles de radiación solar. La conexión entre la base de datos MySQL y la aplicación móvil desarrollada en AndroidStudio permite el acceso a los datos almacenados en tiempo real. La aplicación móvil dispone de secciones específicas para presentar esta información, incluyendo gráficos o mapas que representan los niveles de radiación solar recopilados, tanto en tiempo real como los datos históricos. Para garantizar una actualización constante de los datos, la aplicación móvil ofrece una función de actualización periódica que muestra la información más reciente. Además, permite a los usuarios realizar consultas a la base de datos para acceder a detalles históricos o información detallada sobre los niveles de radiación solar. El proceso de adquisición y almacenamiento de datos implica la conversión de datos binarios a valores hexadecimales y luego a valores decimales antes de su almacenamiento en la base de datos. Este proceso de conversión asegura la adecuada representación y preservación de los datos recolectados. Se han llevado a cabo pruebas de adquisición de datos en diferentes nodos sensores a lo largo de varios días y en diversas condiciones climáticas para validar la efectividad y precisión del sistema.

\bigbreak
En \cite{castillo_herrero_desarrollo_2020} se centra en el desarrollo de una aplicación móvil que utiliza tecnología NFC para gestionar datos de sensores de temperatura asociados a una Base de Datos en un servidor cloud de IoT. El objetivo principal es permitir a los usuarios visualizar y acceder a la información de temperatura de manera rápida y segura. El proceso comienza con la captación de información del mundo real mediante sensores de temperatura asociados a etiquetas NFC. Estos datos se almacenan y gestionan en un servidor dividido en dos secciones: una Base de Datos y un API REST desarrollado con Python y Flask. La aplicación móvil desarrollada en Android Studio se comunica con este servidor para acceder a la información almacenada y presentarla a los usuarios de manera intuitiva. El acceso a la información varía según los roles de los usuarios. Los administradores tienen privilegios para realizar acciones específicas en la Base de Datos, mientras que los usuarios con roles de lectura únicamente pueden visualizar los datos. La lectura y escritura de datos en las etiquetas NFC permiten identificar sensores, registrar temperaturas y acceder al historial de registros. El sistema garantiza la seguridad mediante el uso de contraseñas y roles, asegurando que las acciones de modificación en la Base de Datos se realicen únicamente por usuarios autorizados. Además, se implementa una estrategia de actualización periódica del proyecto en Google Drive para preservar la integridad y continuidad del desarrollo de la aplicación. El TFG enfatiza la importancia de la tecnología NFC en el acceso rápido y seguro a datos, proporcionando una solución eficiente para el seguimiento y control de la temperatura ambiental a través de dispositivos móviles.

\bigbreak
En \cite{krishnan_fuzzy_2020}, se implemento un sistema de riego inteligente basado en lógica difusa que utiliza Internet de las cosas, donde utilizaron la integración de varios sensores, incluyendo el sensor DHT11 para medir humedad y temperatura, además de otros sensores para monitorear diversos parámetros del campo agrícola en tiempo real. Estos datos se muestran en una pantalla LCD y se transmiten al usuario a través de tecnología GSM para facilitar el monitoreo remoto. La infraestructura hardware empleada en este sistema comprende un conjunto de sensores, como el sensor de humedad del suelo, sensor de temperatura, sensor de humedad, sensor de lluvia e incluso un sensor de imagen de hoja de planta. Todos estos sensores están conectados a un controlador Arduino, que actúa como nodo de dispositivo final. Este nodo coordina la recopilación continua de datos de los sensores y los transmite a un nodo coordinador, el cual está conectado al sistema del servidor web a través del bus de datos RS232. Un aspecto destacado es el uso de paneles solares para alimentar el sistema durante la disponibilidad de luz solar, lo que contribuye significativamente a reducir el consumo de energía. La tecnología GSM se emplea para la automatización, minimizando la necesidad de trabajo manual. Este enfoque sostenible a largo plazo permite el control automático del riego y la vigilancia de enfermedades de las plantas. En cuanto a las pruebas realizadas, el sistema fue evaluado comparativamente con métodos tradicionales de riego, como el riego por goteo e inundación manual. Los resultados obtenidos a partir de pruebas comparativas demostraron que el sistema propuesto supera en eficiencia a los métodos tradicionales de riego. En una comparación con el riego por goteo y el riego manual por inundación, el sistema de riego inteligente propuesto bombea agua en un período de 7 horas, mientras que los métodos tradicionales lo hacen en 12 y 20 horas respectivamente. Es decir que el sistema de riego inteligente utiliza el motor durante solo el 9,72\% del tiempo total de riego, en comparación con el 16,67\% y 27,78\% utilizados por el riego por goteo y el riego manual por inundación respectivamente.

\bigbreak
En \cite{kumar_irrigation_2020}, se aborda el desarrollo de un sistema innovador de control de riego basado en la utilización de cuatro tipos de sensores: temperatura, intensidad de luz, humedad del suelo y humedad. Estos sensores desempeñan un papel fundamental en el monitoreo y control del proceso de riego en entornos agrícolas. El objetivo principal de este proyecto es mejorar la eficacia del riego a través de un enfoque que emplea el sistema ANFIS-PEGASIS en una red de sensores inalámbricos (WSN). El proceso de investigación implica la aplicación de un sistema de inferencia difusa (FIS) para la selección óptima de nodos coordinadores (CH) dentro de la red, tomando en cuenta factores como la energía residual y la distancia. Luego, se implementa el protocolo PEGASIS para la recolección eficiente de datos en el sistema de riego, estableciendo una cadena entre los nodos para transmitir la información a la estación base (BS). El enfoque ANFIS se utiliza para la toma de decisiones en el riego, activando automáticamente la bomba de agua y la lámpara según las condiciones detectadas por los sensores. El modelo ANFIS-PEGASIS inicia formando cadenas entre los nodos para la transmisión de datos en la red de sensores, y utiliza FIS para elegir el nodo óptimo como CH. Posteriormente, se inicia la recolección de datos, asegurando una conexión continua y una recolección eficiente incluso en situaciones donde algunos nodos puedan presentar fallas. Los sensores de humedad del suelo y la intensidad de luz, con sus respectivas funciones de pertenencia difusa, aportan datos esenciales para la toma de decisiones del sistema. ANFIS integra 81 reglas difusas para un sistema de riego automático, mapeando estas variables a la activación de la bomba de agua y la lámpara según las condiciones detectadas por los sensores. En términos de pruebas y resultados, se comparó el método ANFIS-PEGASIS con otras técnicas existentes. Se evaluaron parámetros como el rendimiento del retardo de extremo a extremo (E2ED), consumo de energía y rendimiento del sistema. Los resultados de la simulación demostraron que el método propuesto (ANFIS-PEGASIS) superó a otras técnicas existentes en términos de E2ED, rendimiento y sostenibilidad agrícola, proporcionando así una mayor eficacia en el control de riego.

\bigbreak
En \cite{mohammed_intelligent_2021} presenta un sistema de riego inteligente diseñado para la región oriental de Marruecos. La recopilación de datos se lleva a cabo mediante sensores de temperatura, radiación solar y humedad del suelo ubicados estratégicamente en una región agrícola. Los sujetos involucrados incluyen sistemas de riego, condiciones climáticas y cultivos en la región. El proceso del sistema comienza con la adquisición en tiempo real de datos a través de una estación meteorológica y sensores de humedad del suelo conectados a las plantas. Estos datos son esenciales para la toma de decisiones en el sistema. Se implementa un controlador lógico difuso basado en reglas Mamdani que evalúa la humedad del suelo, temperatura y radiación solar para determinar el tiempo óptimo de riego. La lógica difusa se despliega en tres módulos: Fuzzificación, Inferencia y Defuzzificación. La fuzzificación convierte las mediciones digitales de los sensores en variables lingüísticas basadas en funciones de membresía. El módulo de inferencia interpreta estos valores difusos y, mediante reglas si-entonces previamente definidas, asigna valores a la salida utilizando el método Mamdani. El proceso de defuzzificación transforma la salida difusa en una cantidad escalar, permitiendo decisiones precisas sobre el tiempo de riego necesario para mantener la humedad del suelo por encima del 30\%. El sistema se implementa en una placa Arduino Uno R3, utilizando sensores analógicos y digitales para la adquisición de datos. El estudio incluye simulaciones con datos climáticos reales para validar el sistema y examinar su comportamiento en diferentes estaciones del año. Se selecciona un manzano como planta de prueba, y los resultados muestran que el sistema mantiene la humedad del suelo por encima del umbral deseado, eliminando el riesgo de riego insuficiente. El sistema se compara con otros métodos de riego mediante simulaciones, demostrando su eficiencia al consumir la menor cantidad de agua y garantizar condiciones ideales para el crecimiento de las plantas.

\bigbreak
En \cite{benyezza_zoning_2021} se centra en un sistema inteligente de riego basado en tecnología de control difuso e IoT para la conservación de agua y energía en la agricultura. Detalla un sistema de riego inteligente enfocado en maximizar la eficiencia hídrica y energética en la agricultura. Utiliza sensores de humedad del suelo y temperatura ambiental ubicados estratégicamente en zonas del invernadero. Estos sensores forman una red inalámbrica (WSN) que transmite datos a un servidor Node-RED para su procesamiento. El sistema se fundamenta en la tecnología de control difuso para tomar decisiones óptimas de riego. Además, cuenta con una interfaz de usuario desarrollada en Node-RED que facilita el control remoto y la visualización de datos. El proceso de este sistema implica la segmentación del campo en zonas, cada una equipada con nodos que incluyen sensores de humedad del suelo y válvulas solenoides para el riego. Estos datos se recopilan en un nodo central que los transmite a un procesador central (Raspberry Pi 3) donde se implementa un Controlador Lógico Difuso (FLC). Este FLC determina el tiempo óptimo de riego para cada zona basándose en condiciones del suelo y entorno. Las reglas difusas empleadas se extraen de conocimientos expertos y experiencias previas, permitiendo tomar decisiones inteligentes sobre el riego sin la necesidad de un modelo matemático preciso del suelo. El hardware utilizado incluye Arduino Nano, módulo nRF24L01þ para comunicación inalámbrica, sensor DHT22 para medir la temperatura y una batería para la alimentación de los nodos. El sistema se asegura mediante medidas de seguridad en el editor Node-RED para prevenir accesos no autorizados. En las pruebas comparativas con otros tres métodos de riego, el sistema propuesto demostró consumir la menor cantidad de agua y presentar ahorros significativos en el costo de producción. Se observó una reducción sustancial en el consumo energético y un control efectivo de la humedad del suelo, favoreciendo el crecimiento de las plantas. Se destacó un ahorro del 46,81\% en el costo de producción en comparación con el primer método de riego, y ahorros aún más significativos en comparación con los otros dos métodos tradicionales. Este sistema inteligente de riego, al incorporar la irrigación por zonas, control difuso, comunicación inalámbrica y control remoto, demostró ser altamente efectivo, ofreciendo mejoras superiores al 80\% en comparación con las estrategias de riego más comúnmente utilizadas.

\bigbreak
En \cite{widyawati_fuzzy_2022}, se desarrollo un diseño de lógica difusa para controlar la duración del tiempo de riego en un invernadero. Este diseño emplea datos obtenidos a partir de sensores de humedad del suelo y el sensor DHT22 para monitorizar la humedad y temperatura, fundamentales en el proceso de riego en invernaderos. Los sujetos de estudio son los invernaderos mismos y las plantas que se riegan en este entorno. El objetivo central se centra en el diseño y la implementación de un controlador de lógica difusa (FLC) tipo Mamdani, integrado en una placa NodeMCU ESP8266. Este sistema utiliza los datos de los sensores como entrada para determinar la duración óptima del riego. La propuesta destaca por su conectividad IoT (Internet de las cosas), lo que se espera permita tomar decisiones precisas de riego y optimizar el uso del agua en invernaderos. El proceso de investigación se inicia con un análisis detallado de los problemas y requisitos del sistema, seguido por la recopilación de datos a través de investigaciones y literatura especializada. La fase de diseño describe el esquema y flujo de trabajo del sistema, y conlleva la implementación del hardware y la carga del programa según el diseño establecido. Las pruebas y evaluaciones posteriores comparan la lógica difusa en NodeMCU ESP8266 con el software MATLAB para validar su desempeño. La lógica difusa se estructura en etapas de fuzzificación, inferencia basada en reglas y defuzzificación para determinar la duración del riego a partir de la humedad y temperatura del suelo. Los valores de entrada se convierten en datos lingüísticos difusos, aplicando funciones de pertenencia difusas y reglas de conexión para determinar la duración óptima del riego. La implementación del sistema integra una puerta de enlace y nodos de sensores, donde la lectura y procesamiento de datos se realizan mediante lógica difusa en NodeMCU ESP8266, programada en lenguaje C con Arduino IDE. El artículo detalla los componentes de hardware empleados, entre los que se encuentran NodeMCU ESP8266, sensores de humedad del suelo, sensor DHT22, módulo TP4056, batería de ion de litio 18650, Raspberry Pi y cables jumper. Los resultados de las pruebas comparativas entre la lógica difusa basada en NodeMCU ESP8266 y la de MATLAB indican un error promedio del 0.59\%. Esto sugiere que el controlador de lógica difusa con el método Mamdani integrado en el NodeMCU ESP8266 ofrece una alta precisión para controlar la duración del riego en invernaderos.

\bigbreak
En \cite{haiyan_intelligent_2022} se destaca una plataforma avanzada de gestión de recursos hídricos para agricultura que integra datos de temperatura, humedad y crecimiento de plantas mediante sensores y cámaras digitales. Esta plataforma emplea técnicas de procesamiento de imágenes digitales y lógica difusa para analizar la información de crecimiento de cultivos, evaluar la necesidad de riego y detectar la escasez de agua en las plantas. El proceso de adquisición de datos inicia con la detección precisa del suelo y su entorno mediante sensores especializados, como el sensor de humedad del suelo SM3002B y el sensor de temperatura y humedad del aire SHT30. Estos dispositivos capturan información en tiempo real sobre las condiciones del suelo y el ambiente de cultivo. Posteriormente, esta información se transmite de forma inalámbrica a un servidor central utilizando un protocolo específico para su procesamiento. El procesamiento central se lleva a cabo mediante microcontroladores como la microcomputadora de un solo chip modelo STC15F2K60S2 y el chip MSP430. Estos componentes realizan la conversión y procesamiento de los datos recolectados. Además, la plataforma cuenta con una pantalla LCD de 320 * 240, que muestra información ambiental en tiempo real. Este sistema emplea algoritmos de control difuso y control PID para ajustar automáticamente el riego según la información recopilada. La alimentación de la plataforma se sustenta en energía solar, lo que garantiza su funcionamiento sostenible y minimiza su impacto ambiental. En relación con la estabilidad y confiabilidad de la red de comunicación, los autores realizaron un experimento que se enfoca en comparar las redes de transmisión punto a punto y las redes de múltiples saltos. Este estudio destaca la importancia de una adecuada distribución de nodos y la gestión de la distancia entre ellos para mantener una comunicación eficaz y estable. Estos resultados son fundamentales en sistemas de riego, donde la fiabilidad de la comunicación es esencial para el control efectiva del sistema.

\bigbreak
En \cite{noauthor_fuzzy_2023} presenta un sistema de control de riego basado en lógica difusa, destinado a optimizar la producción agrícola mediante el uso de datos meteorológicos y datos de sensores especializados, como sensores de humedad del suelo, temperatura y humedad. Este sistema se integra con sensores en robots móviles, datos meteorológicos y una aplicación móvil, los cuales monitorean constantemente variables clave como la humedad del suelo, la humedad y la temperatura. El proceso se inicia con la recopilación y envío de datos de estos sensores a un servidor en la nube. Estos datos se contrastan con una base de conocimientos previamente establecida sobre las necesidades específicas de distintos cultivos. Utilizando reglas difusas, el sistema toma decisiones sobre el funcionamiento del motor del sistema de riego, basándose en las condiciones medidas por los sensores. Estas reglas son declaraciones lógicas que relacionan las entradas de los sensores con la activación o desactivación del motor, dependiendo de las necesidades hídricas del cultivo. El sistema se apoya en sensores como el DTH11 para la medición de temperatura y humedad, así como sensores de humedad del suelo. Además, utiliza hardware como Arduino Mega, controlador ICL2930 y microcontrolador ESP8266, capaz de comunicarse por WiFi, para el procesamiento y la comunicación de datos. La contribución principal de este estudio radica en el desarrollo de un sistema de riego inteligente que emplea robots terrestres equipados con estos sensores y un modelo de inferencia difusa para tomar decisiones sobre el motor del aspersor. La aplicación móvil proporciona visualización de parámetros del sensor y control remoto del sistema de riego, permitiendo al agricultor supervisar y en algunos casos influir en el proceso de toma de decisiones. Las pruebas realizadas para evaluar el sistema incluyeron diversos aspectos. Desde la evaluación de las reglas difusas hasta la precisión en la toma de decisiones para activar o desactivar el motor de riego en varios escenarios. De manera destacada, se logró una precisión del 97\% en la toma de decisiones durante la evaluación, lo que significa que, de 1079 casos de prueba, el sistema de inferencia difusa evaluó correctamente la salida para 1046 muestras.

\bigbreak
El estudio presentado en \cite{neugebauer_fuzzy_2023} se centra en el análisis de un sistema de riego inteligente basado en lógica difusa. Este análisis combina datos teóricos, mediciones reales y simulaciones computacionales. Utiliza el método de elementos finitos (FEM) para modelar la propagación del agua en el suelo, permitiendo simular la irrigación. Se evalúan múltiples variables como la temperatura ambiente, la humedad del suelo y la hora del día, las cuales controlan la intensidad del riego a través de un sistema de control lógico difuso (FLC). El proceso de investigación abarca la creación de modelos tanto teóricos como prácticos para comparar la eficiencia de los sistemas de riego convencionales con aquellos controlados por lógica difusa. Mediante un modelo 2D, se logra simular y comparar distintos sistemas de riego, facilitando la elección del más eficiente para suministrar agua a las raíces de las plantas con el menor consumo posible. El uso de un controlador de lógica difusa, basado en variables como temperatura, humedad del suelo y hora del día, permite determinar el tiempo y la cantidad óptima de riego. Las pruebas incluyen la ubicación de sensores de humedad del suelo a diferentes profundidades y su impacto en el consumo de agua. Se valida el modelo considerando el consumo de agua y el tiempo de riego para un tipo específico de planta (césped) con una profundidad de raíz determinada. Los resultados resaltan la eficiencia del controlador de lógica difusa en la gestión del riego, demostrando su capacidad para reducir el consumo de agua en la producción vegetal y contribuir a la gestión sostenible de recursos hídricos limitados. Además, se realizaron pruebas comparativas entre sistemas de riego convencionales y aquellos controlados por lógica difusa bajo diversas condiciones y distribuciones de variables. En estas pruebas se observó que el sistema controlado por lógica difusa utilizó un 13\% menos de agua en comparación con el sistema convencional. Sin embargo, en una simulación con distribuciones diferentes de variables, aunque con las mismas reglas, se notó un aumento considerable en el consumo de agua por parte del sistema controlado por lógica difusa.

\bigbreak
En \cite{ramos_galindo_diseno_2023} se basa en el desarrollo de una aplicación móvil utilizando el framework Flutter para el monitoreo de humedad del suelo. Este sistema se complementa con un conjunto de sensores controlados por un microcontrolador ESP32 y hace uso de Firebase como base de datos en tiempo real. La arquitectura utilizada implica el uso de Flutter como framework principal para la creación de la aplicación móvil, permitiendo el desarrollo de interfaces intuitivas y la visualización de datos en tiempo real provenientes de sensores. El microcontrolador ESP32 se elige por su capacidad integral para manejar las necesidades del proyecto, incluyendo la adquisición de datos analógicos de los sensores de humedad. En cuanto a la base de datos y autenticación, Firebase es seleccionado por su capacidad de mantener operativa una base de datos en tiempo real las 24/7. Esto permite almacenar y gestionar los datos obtenidos por los sensores de manera eficiente y accesible. El proceso inicia con la adquisición de datos a través de sensores de humedad conectados al microcontrolador ESP32, que capturan información en tiempo real sobre las condiciones del suelo. Estos datos se transmiten a la base de datos en Firebase para su procesamiento y almacenamiento. La aplicación móvil desarrollada en Flutter permite a los usuarios visualizar esta información de manera intuitiva y tomar decisiones basadas en las necesidades específicas de cada cultivo. La arquitectura general del sistema comprende el uso de sensores conectados al microcontrolador ESP32 para la adquisición de datos, Firebase para el almacenamiento en la nube, y Flutter para la interfaz de usuario y visualización de datos en tiempo real. Este sistema busca proporcionar una solución integral para el monitoreo de humedad del suelo, permitiendo a agricultores, horticultores y entusiastas del cultivo acceder fácilmente a información valiosa para mejorar el cuidado de sus cultivos.
