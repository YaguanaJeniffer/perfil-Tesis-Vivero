\subsection{Antecedentes investigativos}
Luego de un análisis en repositorios digitales, plataformas académicas y
diversas bases de datos científicas como: IEEE SPRINGER LINK, GOOGLE SCHOLAR y
LA UNIVERSIDAD TÉCNICA DE AMBATO, se ha recopilado una extensa variedad de
información. Asimismo se limitó la búsqueda al período comprendido entre 2018 y
2023.

\bigbreak
Sin embargo, para asegurar la relevancia de los datos recopilados se
aplicaron criterios de inclusión y exclusión detallados en la Tabla
\ref{tabla_1}, descartando aquellos artículos que no cumplían con estos
criterios.

\begin{table}[h]
    \centering
    \caption{Criterios de inclusión y exclusión}
    \label{tabla_1}
    \begin{tabular}{|p{6cm}|p{6cm}|}
    \hline
    \textbf{Inclusión} & \textbf{Exclusión} \\
    \hline
    Artículos escritos en inglés y español. & Artículos escritos en otros idiomas que no sean en inglés y español. \\
    Artículos publicados entre los años 2018 – 2023. & Artículos publicados antes de 2018. \\
    Tesis y Artículos & Libros, páginas web y foros \\
    Artículo relacionados con monitoreo y control de riego usando Lógica Difusa. & Artículos no relevantes con la temática principal. \\
    Artículos disponibles en bibliotecas reconocidas o académicas que se enumeran en la sección de Base de datos. & Artículos disponibles en otras bibliotecas que no se enumeran en la sección de Base de datos. \\
    \hline
    \end{tabular}
    \end{table}
    

Después de aplicar el proceso de exclusión, se identificaron un total de 21
artículos relacionados con sistemas de riego que involucran la lógica difusa.
Aunque estos artículos no coinciden directamente con el objetivo específico de
esta investigación, se resaltan a continuación los trabajos más relevantes para
el desarrollo del proyecto.

\bigbreak
En \cite{lema_holguin_implementacion_2018} se implementó un sistema automatizado de riego por goteo. Se utilizó una configuración de sensores para monitorear la humedad del sustrato, la conductividad eléctrica del suelo y el pH del agua, ubicados en contenedores de cultivo individuales con 24 plantas de tomate en cada uno. El proyecto presentó una solución para controlar y optimizar las condiciones de cultivo. Se implementó un invernadero que proporcionaba microclimas controlados para mitigar los impactos climáticos externos y mantener un entorno óptimo para el desarrollo de las plantas. Los sensores de temperatura, humedad, pH y conductividad eléctrica del sustrato se conectaron a dos controladores difusos tipo Mamdani, encargados de regular el riego y la administración de nutrientes en base a las lecturas obtenidas. En cuanto al hardware utilizado, se combinaron varios componentes específicos. Entre ellos se encuentran sensores como el Sensor Con-BTA y YL-68 para mediciones de pH y humedad, el Sensor Digital de Temperatura DS18B20, junto con tarjetas Arduino Mega y Leonardo con Ethernet Shield para control y comunicación de datos. Se integraron bombas de agua, electroválvulas y otros sensores, cada uno asignado a funciones específicas dentro del sistema de riego por goteo automatizado. El diseño del sistema incluyó la definición de reglas difusas para los controladores, los cuales ejecutaban acciones en función de las lecturas de los sensores. Se realizaron mediciones periódicas mediante sensores y dispositivos de monitoreo para mantener las condiciones óptimas de cultivo. Los tiempos de respuesta de los controladores se mantuvieron dentro de los 5 segundos, comparados con tiempos obtenidos en Matlab. Esta combinación facilitó la comunicación de datos y la gestión de la base de datos en phpMyAdmin, permitiendo un análisis de las variables del proceso. Sin embargo la presentación de datos numéricos que demuestren la estabilidad y precisión del sistema bajo diversas condiciones ambientales brindaría una comprensión más detallada de su rendimiento práctico.

\bigbreak
En \cite{alcivar_dominguez_sistema_2018} se implementó un sistema automatizado de riego basado en sensorización en los cultivos de ciclo corto. Este se compone de una variedad de componentes electrónicos clave. Entre ellos se encuentran sensores como el Módulo HL 69, el Módulo YL-83 y el DHT22, diseñados para recopilar datos precisos sobre la humedad del suelo, humedad ambiental y temperatura. El Arduino Mega se utilizó como el núcleo de control, encargado de coordinar la información de los sensores y tomando decisiones para el riego eficiente de los cultivos. Por otro lado, el software desarrollado para este sistema se basa en la plataforma de Visual Studio Community 2015. Esta elección el autor respalda por la versatilidad que ofrece, especialmente a través de Windows Form que facilitó el diseño y la presentación de consultas y reportes generados en el sistema. La integración de una base de datos MySQL-2012 le permitió la creación de archivos de tipo app config para manejar los parámetros de conexión al servidor, facilitando la lectura y almacenamiento de información para el sistema. La aplicación de escritorio se llevó a cabo bajo la metodología Cascada asegurando una secuencia metódica desde el análisis de necesidades hasta la implementación física y la validación del sistema en el terreno. Los resultados en la reducción del consumo de agua fueron de un 37.27\%, un aumento del 7.53\% en la productividad de los cultivos y un eficiente control de humedad del suelo. Si bien la elección de la metodología cascada para el desarrollo del software brindó una estructura metódica y secuencial en el proyecto de implementación del sistema de riego, se sugiere que podría haberse complementado con metodologías ágiles como Scrum o Kanban.

\bigbreak
En \cite{salazar_irrigation_2013} desarrolló un sistema de riego a través de Agentes Inteligentes usando la Tecnología Arduino. La arquitectura que los autores propusieron se basa en agentes inteligentes que interactúan entre sí y con el entorno, utilizando una combinación de lógica difusa y el modelo BDI (Believe, Desire, Intention). El proceso inicia con la placa Arduino MEGA, seleccionada por el autor por su capacidad de conectividad y control, en comunicación con diversos sensores como el sensor de humedad del suelo, y actuadores que controlan el riego. Los sensores recopilan datos sobre la humedad del suelo, que son procesados por un Agente de Campo y luego enviados al Agente de Control. La lógica difusa la implementaron para utilizar un sistema MISO (entrada múltiple, salida única), datos de la humedad del suelo y su traducción en variables lingüísticas difusas. Esta arquitectura multiagente se desarrolla sobre la plataforma JADE, siguiendo el estándar FIPA para sistemas multiagente. La interacción entre Arduino y la computadora se realiza mediante una conexión serial, y se ha desarrollado una interfaz de software que muestra detalles de los sensores y el estado del sistema de riego, con modos manual y automático. Las pruebas del sistema se llevaron a cabo en un proyecto de guayabas taiwanesas, demostrando que el sistema de riego basado en agentes inteligentes permitió un monitoreo en tiempo real de las necesidades de agua de cada cultivo, adaptándose a las condiciones específicas del suelo. Sin embargo, la omisión de detalles específicos sobre el tipo de hardware utilizado más allá de la placa Arduino MEGA limita la comprensión completa del sistema.

\bigbreak
En \cite{hasan_implementation_2018} los autores se centran en la gestión óptima de la humedad del suelo. Este sistema automatizado utiliza sensores especializados para monitorear constantemente el contenido de humedad en el suelo, enviando datos al sistema cada segundo. El sistema emplea lógica difusa para calcular el nivel de control requerido. Esto activa un método de control que ajusta una válvula en la línea de suministro de agua, permitiendo el flujo desde un tanque elevado estático hacia el campo. El proceso de riego se adaptaba dinámicamente según los niveles de humedad hasta al alcanzar el nivel óptimo. Este modelo de riego automatizado basado en un algoritmo difuso se encargaba de clasificar la humedad en seis clases: 'Muy bajo, bajo, medio, preciso, superior y desbordado'. Además, utiliza microcontroladores, motores y bombas para su funcionamiento. Los resultados que obtuvieron los autores de las pruebas comparativas entre el método de riego tradicional y este modelo propuesto fue una reducción del 12.3\% en el uso de agua. De esta manera sería favorable si se incluyeran detalles técnicos más específicos sobre los microcontroladores, motores y bombas utilizados, así como información detallada sobre la metodología de las pruebas comparativas entre el método tradicional de riego y el modelo propuesto.

\bigbreak
En \cite{munir_intelligent_2019}, desarrollaron un sistema de riego inteligente que se apoya en la lógica difusa y la tecnología blockchain para gestionar eficientemente el riego de las plantas. Utilizando datos provenientes de sensores que monitorean variables como temperatura, humedad, intensidad de luz, nivel de humedad del suelo y estado de la bomba de agua, este sistema toma decisiones en tiempo real respecto al riego de las plantas. Este sistema se compone de hardware como Arduino-UNO R3, Bread Board, sensores DHT-11 de temperatura y humedad, y sensor YL-69 de humedad del suelo. Estos dispositivos recopilan información crucial sobre las condiciones ambientales y del suelo, vital para determinar las necesidades hídricas de las plantas. La conectividad inalámbrica se logra mediante un módulo Wi-Fi ESP 8266-01, facilitando la transmisión de datos entre sensores y el servidor central. La aplicación móvil para Android permitió a los usuarios monitorear y controlar remotamente el sistema de riego, configurar horarios de riego, verificar la humedad del suelo y recibir recomendaciones personalizadas para el cuidado de las plantas. La metodología empleada se basa en Fuzzy Logic y Blockchain. La lógica difusa facilita la toma de decisiones sobre las necesidades de riego, mientras que la tecnología blockchain asegura la conectividad segura de dispositivos en un entorno IoT, permitiendo el acceso exclusivo a dispositivos confiables. El resultado que obtuvo el autor de las pruebas experimentales, la mayoría alcanzaron un 100\% de efectividad, es decir que la precisión general fue del 95.83\%. Sin embargo podría haberse reforzarse estos resultados con una mayor descripción de cómo se realizaron estas pruebas, qué variables se consideraron y cómo se evaluó la efectividad.

\bigbreak
En \cite{al-ali_iot-solar_2019}, se menciona un sistema de riego inteligente alimentado por energía solar para la agricultura. El sistema recopila múltiples tipos de datos incluyendo lecturas de humedad y temperatura del suelo, provenientes de sistemas de riego y componentes electrónicos. Los objetivos principales señalados por el autor se centran en el monitoreo remoto, control de humedad, temperatura, energía solar y riego automático en entornos agrícolas. Para poder lograr esto, desarrollaron algoritmos de control y lógica difusa, los cuales les permitió medir la eficiencia del riego, la adaptabilidad a condiciones climáticas variables y el consumo de energía solar. El sistema se basa en un controlador de National Instruments, myRIO, con un procesador ARM de doble núcleo y una matriz FPGA, conectado a una variedad de sensores como humedad del suelo, humedad y temperatura, sensor de flujo, interruptores de flotador y actuadores como bombas de achique y de diafragma. El proceso de control se desglosa en tres modos: control local, monitoreo y control móvil, y control basado en lógica difusa. El último modo implica tomar decisiones sobre el encendido o apagado de las bombas basándose en lecturas de sensores y un algoritmo de control difuso. El sistema de riego inteligente propuesto se diseñó con la premisa de ser accesible y controlable desde cualquier ubicación y en cualquier momento, ofreciendo al usuario la capacidad de asumir el control total del sistema cuando así lo desee. Para lograr esta funcionalidad utilizaron un servidor web remoto que se respalda con una base de datos en lenguaje de consulta estructurado (SQL). Esta base de datos se encarga de almacenar de manera organizada y estructurada las mediciones provenientes de los sensores utilizados en el sistema. Ademas, se utilizó un controlador de sistema en un chip de placa única con conectividad WiFi y conexiones a una celda solar para leer sensores de campo y emitir señales de comando para operar las bombas de riego. Se llevaron a cabo pruebas para verificar el funcionamiento del sistema en diferentes condiciones de humedad y temperatura del suelo. No obstante, sería útil incorporar detalles de resultados numéricos o métricas específicas como la reducción del consumo de agua o el porcentaje de mejora en la eficiencia del riego.

\bigbreak
En \cite{orozco_jaramillo_diseno_2019} se ha diseñado e implementado una red de sensores para monitorear los niveles de radiación solar en la ciudad de Loja, se destaca la utilización de sensores GUVA-S12SD, Raspberry Pi 3B y una aplicación móvil desarrollada en Android Studio para llevar a cabo la medición y visualización de estos niveles. El proceso de adquisición de datos se realiza a través de sensores especializados, como el sensor GUVA-S12SD, dedicados a monitorear los niveles de radiación solar. Estos sensores capturan la información y la envían a una unidad central, la Raspberry Pi, que actúa como concentrador de datos. Posteriormente, la Raspberry Pi recibe, procesa y almacena los datos provenientes de los sensores en una base de datos MySQL. La conexión entre la base de datos MySQL y la aplicación móvil desarrollada en Android Studio permitió el acceso a los datos almacenados en tiempo real. La aplicación móvil disponía de secciones específicas para presentar esta información, incluyendo gráficos o mapas que representan los niveles de radiación solar recopilados, tanto en tiempo real como los datos históricos. Además, permitía a los usuarios realizar consultas a la base de datos para acceder a detalles históricos o información detallada sobre los niveles de radiación solar. El proceso de adquisición y almacenamiento de datos implica la conversión de datos binarios a valores hexadecimales y luego a valores decimales antes de su almacenamiento en la base de datos. Los autores llevaron a cabo pruebas de adquisición de datos en diferentes nodos sensores a lo largo de varios días y en diversas condiciones climáticas para validar la efectividad y precisión del sistema. De este modo podría haber señalado ha detalle los procesos de calibración o validación de los sensores utilizados para garantizar la precisión de las mediciones.

\bigbreak
En \cite{castillo_herrero_desarrollo_2020} se centró en el desarrollo de una aplicación móvil que utilizó tecnología NFC para gestionar datos de sensores de temperatura asociados a una Base de Datos en un servidor cloud de IoT. El objetivo principal que señalaron los autores es permitir a los usuarios visualizar y acceder a la información de temperatura de manera rápida y segura. El proceso comienza con la captación de información a través de sensores de temperatura asociados a etiquetas NFC. Estos datos se almacenan y gestionan en un servidor dividido en dos secciones: una Base de Datos y un API REST desarrollado con Python y Flask. La aplicación móvil desarrollada en Android Studio se comunica con este servidor para acceder a la información almacenada y presentarla a los usuarios de manera intuitiva. El acceso a la información varía según los roles de los usuarios. La lectura y escritura de datos en las etiquetas NFC permiten identificar sensores, registrar temperaturas y acceder al historial de registros. El sistema garantiza la seguridad mediante el uso de contraseñas y roles, asegurando que las acciones de modificación en la Base de Datos se realicen únicamente por usuarios autorizados. Además, se implementa una estrategia de actualización periódica del proyecto en Google Drive para preservar la integridad y continuidad del desarrollo de la aplicación. Sin embargo podría haber sugerido la inclusión de detalles adicionales sobre la eficacia y precisión de la tecnología NFC utilizada para la captación de datos de los sensores.

\bigbreak
En \cite{krishnan_fuzzy_2020} implementaron un sistema de riego inteligente basado en lógica difusa que utiliza Internet de las cosas, donde utilizaron la integración de varios sensores, incluyendo el sensor DHT11 para medir humedad y temperatura. Además utilizaron otros sensores para monitorear diversos parámetros del campo agrícola en tiempo real. Estos datos se muestran en una pantalla LCD y se transmiten al usuario a través de tecnología GSM para facilitar el monitoreo remoto. La infraestructura hardware empleada en este sistema comprende un conjunto de sensores, como el sensor de humedad del suelo, sensor de temperatura, sensor de humedad, sensor de lluvia e incluso un sensor de imagen de hoja de planta. Todos estos sensores están conectados a un controlador Arduino, que actúa como nodo de dispositivo final. Un aspecto destacado es el uso de paneles solares para alimentar el sistema durante la disponibilidad de luz solar. La tecnología GSM se empleó para la automatización dando como resultado la disminución de la necesidad de trabajo manual. Los autores realizaron pruebas al sistema, donde se evaluó de manera comparativa los métodos tradicionales de riego, el riego por goteo e inundación manual. En una comparación con el riego por goteo y el riego manual por inundación, se destacó que el sistema de riego inteligente propuesto bombea agua en un período de 7 horas, mientras que los métodos tradicionales lo hacen en 12 y 20 horas respectivamente. Es decir que el sistema de riego inteligente utiliza el motor durante solo el 9,72\% del tiempo total de riego, en comparación con el 16,67\% y 27,78\% utilizados por el riego por goteo y el riego manual por inundación respectivamente. En mi opinión podrían haberse centrado en profundizar en la comparación de la eficiencia del sistema propuesto con los métodos tradicionales de riego, proporcionando más detalles sobre la precisión de las mediciones de humedad y temperatura realizadas por los sensores.

\bigbreak
En \cite{kumar_irrigation_2020} abordo el desarrollo de un sistema de control de riego basado en la utilización de cuatro tipos de sensores: temperatura, intensidad de luz, humedad del suelo y humedad. Estos sensores fueron necesarios para en el monitoreo y control del proceso de riego en entornos agrícolas. El objetivo principal que presentaron los autores de este proyecto es mejorar la eficacia del riego a través de un enfoque que emplea el sistema ANFIS-PEGASIS en una red de sensores inalámbricos (WSN). El proceso de investigación implicó la aplicación de un sistema de inferencia difusa para la selección óptima de nodos coordinadores (CH) dentro de la red, tomando en cuenta factores como la energía residual y la distancia. Luego se implementó el protocolo PEGASIS para la recolección eficiente de datos en el sistema de riego, estableciendo una cadena entre los nodos para transmitir la información a la estación base (BS). El enfoque ANFIS se utilizó para la toma de decisiones en el riego, activando automáticamente la bomba de agua y la lámpara según las condiciones detectadas por los sensores. Posteriormente se inició con la recolección de datos, asegurando una conexión continua y una recolección eficiente. ANFIS integró 81 reglas difusas para el sistema de riego automático, de manera que mapeaba estas variables para la activación de la bomba de agua y la lámpara según las condiciones detectadas por los sensores. En términos de pruebas y resultados los autores compararon el método ANFIS-PEGASIS con otras técnicas existentes. Se evaluaron parámetros como el rendimiento del retardo de extremo a extremo (E2ED), consumo de energía y rendimiento del sistema. Asimismo los autores mencionan que los resultados de la simulación demostraron que el método propuesto (ANFIS-PEGASIS) superó a otras técnicas existentes en términos de E2ED, rendimiento y sostenibilidad agrícola. Sería muy útil para el estudio si se incluyera la comparación detallada que se realizó con otras técnicas.

\bigbreak
En \cite{mohammed_intelligent_nodate} presentó un sistema de riego inteligente diseñado para la región oriental de Marruecos, en el cual se llevó a cabo la recopilación de datos mediante sensores estratégicamente ubicados en una región agrícola. Los datos se adquirieron en tiempo real a través de una estación meteorológica y sensores de humedad del suelo conectados a las plantas. Se implementó un controlador lógico difuso basado en reglas Mamdani que evaluaba la humedad del suelo, temperatura y radiación solar para determinar el tiempo óptimo de riego. El sistema se complementó con una placa Arduino Uno R3, permitiendo la adquisición de datos mediante sensores analógicos y digitales. Se realizaron simulaciones con datos climáticos reales para validar el sistema y analizar su comportamiento en diferentes estaciones del año. Se seleccionó un manzano como planta de prueba, y los resultados mostraron que el sistema mantuvo la humedad del suelo por encima del umbral deseado, evitando el riesgo de riego insuficiente. A pesar de ello, sería útil profundizar en la descripción técnica del hardware utilizado para mejorar la comprensión del sistema.

\bigbreak
En \cite{benyezza_zoning_2021} se centró en maximizar la eficiencia hídrica y energética en la agricultura. Se implementó un sistema de riego basado en tecnología de control difuso e IoT, el cual utilizó sensores de humedad del suelo y temperatura ambiental en zonas estratégicas del invernadero. Estos sensores formaron una red inalámbrica que transmitió datos a un servidor Node-RED para su procesamiento. El proceso se fundamentó en la tecnología de control difuso para tomar decisiones óptimas de riego, sin requerir un modelo matemático preciso del suelo. Se implementó un Controlador Lógico Difuso (FLC) en un nodo central, basándose en reglas difusas extraídas de conocimientos expertos y experiencias previas. El hardware empleado incluyó Arduino Nano, módulo nRF24L01þ para comunicación inalámbrica, sensor DHT22 para medir la temperatura y una batería para la alimentación de los nodos. Además, se aplicaron medidas de seguridad en el editor Node-RED para asegurar el sistema contra accesos no autorizados. En las pruebas comparativas con otros métodos de riego, los autores observaron que el sistema propuesto logró un consumo reducido de agua, significativos ahorros en el costo de producción, una disminución en el consumo energético y un control efectivo de la humedad del suelo, favoreciendo el crecimiento de las plantas. Se destacó que este sistema inteligente de riego, al incorporar la irrigación por zonas, control difuso, comunicación inalámbrica y control remoto, ofreció mejoras superiores al 80\% en comparación con estrategias de riego tradicionales. Sin embargo, podría haber estado en evaluación y adaptaciones continuas del sistema para abordar una gama más amplia de condiciones ambientales. Al ampliar la consideración de diversos escenarios, podría haberse optimizado aún más el rendimiento del sistema, aumentando su versatilidad y eficacia en una variedad más amplia de entornos agrícolas.

\bigbreak
En \cite{widyawati_fuzzy_2022}, se desarrolló un diseño de lógica difusa para controlar la duración del tiempo de riego en un invernadero. Este diseño empleo datos obtenidos a partir de sensores de humedad del suelo y el sensor DHT22 para monitorizar la humedad y temperatura. El objetivo central se centró en el diseño y la implementación de un controlador de lógica difusa (FLC) tipo Mamdani, integrado en una placa NodeMCU ESP8266. Este sistema utilizó los datos de los sensores como entrada para determinar la duración óptima del riego. El proceso inicio con un análisis detallado de los problemas y requisitos del sistema, seguido por la recopilación de datos a través de investigaciones y literatura especializada. La fase de diseño describe el esquema y flujo de trabajo del sistema, y conlleva la implementación del hardware y la carga del programa según el diseño establecido. Las pruebas y evaluaciones posteriores comparan la lógica difusa en NodeMCU ESP8266 con el software MATLAB para validar su desempeño. La lógica difusa se estructuraba en etapas de fuzzificación, inferencia basada en reglas y defuzzificación para determinar la duración del riego a partir de la humedad y temperatura del suelo. Los valores de entrada se convierten en datos lingüísticos difusos, aplicando funciones de pertenencia difusas y reglas de conexión para determinar la duración óptima del riego. La implementación del sistema integró una puerta de enlace y nodos de sensores, donde la lectura y procesamiento de datos se realizan mediante lógica difusa en NodeMCU ESP8266, programada en lenguaje C con Arduino IDE. El artículo detalla los componentes de hardware empleados, entre los que se encuentran NodeMCU ESP8266, sensores de humedad del suelo, sensor DHT22, módulo TP4056, batería de ion de litio 18650, Raspberry Pi y cables jumper. Los resultados de las pruebas comparativas entre la lógica difusa basada en NodeMCU ESP8266 y la de MATLAB indican un error promedio del 0.59\%. Esto sugirió que el controlador de lógica difusa con el método Mamdani integrado en el NodeMCU ESP8266 ofrece una alta precisión para controlar la duración del riego en invernaderos. Sin embargo, se podría considerar la inclusión de análisis y resultados adicionales en términos de la robustez del sistema frente a variaciones extremas en las condiciones ambientales. Esto podría proporcionar una visión más completa de la aplicabilidad del sistema en diferentes contextos agrícolas y en diversas situaciones climáticas.

\bigbreak
En \cite{haiyan_intelligent_2022} se destacó una plataforma avanzada de gestión de recursos hídricos para agricultura que integra datos de temperatura, humedad y crecimiento de plantas mediante sensores y cámaras digitales. Esta plataforma emplea técnicas de procesamiento de imágenes digitales y lógica difusa para analizar la información de crecimiento de cultivos, evaluar la necesidad de riego y detectar la escasez de agua en las plantas. El proceso de adquisición de datos inicio con la detección precisa del suelo y su entorno mediante sensores especializados, como el sensor de humedad del suelo SM3002B y el sensor de temperatura y humedad del aire SHT30. Estos dispositivos capturaron información en tiempo real sobre las condiciones del suelo y el ambiente de cultivo. Posteriormente, esta información se transmite de forma inalámbrica a un servidor central utilizando un protocolo específico para su procesamiento. El procesamiento central se lleva a cabo mediante microcontroladores como la microcomputadora de un solo chip modelo STC15F2K60S2 y el chip MSP430. Estos componentes realizan la conversión y procesamiento de los datos recolectados. Además, la plataforma cuenta con una pantalla LCD de 320 * 240, que muestra información ambiental en tiempo real. Este sistema emplea algoritmos de control difuso y control PID para ajustar automáticamente el riego según la información recopilada. La alimentación de la plataforma se sustenta en energía solar. En relación con la estabilidad y confiabilidad de la red de comunicación, los autores realizaron un experimento que se enfoca en comparar las redes de transmisión punto a punto y las redes de múltiples saltos. Este estudio destacó la importancia de una adecuada distribución de nodos y la gestión de la distancia entre ellos para mantener una comunicación eficaz y estable. Estos resultados son fundamentales en sistemas de riego, donde la fiabilidad de la comunicación es esencial para el control efectiva del sistema.

\bigbreak
En \cite{noauthor_fuzzy_2023} presentó un sistema de control de riego basado en lógica difusa, destinado a optimizar la producción agrícola mediante el uso de datos meteorológicos y datos de sensores especializados, como sensores de humedad del suelo, temperatura y humedad. Este sistema se integra con sensores en robots móviles, datos meteorológicos y una aplicación móvil, los cuales monitorean constantemente variables clave como la humedad del suelo, la humedad y la temperatura. El proceso se inició con la recopilación y envío de datos de estos sensores a un servidor en la nube. Estos datos se almacenaron en una base de conocimientos previamente establecida sobre las necesidades específicas de distintos cultivos. Los autores utilizaron reglas difusas para la toma decisiones sobre el funcionamiento del motor del sistema de riego, basándose en las condiciones medidas por los sensores. El sistema se apoya en sensores como el DTH11 para la medición de temperatura y humedad, así como sensores de humedad del suelo. Además, utiliza hardware como Arduino Mega, controlador ICL2930 y microcontrolador ESP8266, capaz de comunicarse por WiFi, para el procesamiento y la comunicación de datos. La aplicación móvil proporciono visualización de parámetros del sensor y control remoto del sistema de riego, permitiéndole al agricultor supervisar y en algunos casos influir en el proceso de toma de decisiones. Las pruebas que realizaron los autores para evaluar el sistema incluyeron diversos aspectos. Desde la evaluación de las reglas difusas hasta la precisión en la toma de decisiones para activar o desactivar el motor de riego en varios escenarios. Se menciono una precisión del 97\% en la toma de decisiones durante la evaluación, lo que significa que, de 1079 casos de prueba, el sistema de inferencia difusa evaluó correctamente la salida para 1046 muestras. Sin embargo, seria de mayor utilidad la profundización en la discusión sobre la precisión del sistema en condiciones extremas.

\bigbreak
El estudio presentado en \cite{neugebauer_fuzzy_2023} se centró en el análisis de un sistema de riego inteligente basado en lógica difusa. Este análisis combinó datos teóricos, mediciones reales y simulaciones computacionales. Utilizaron el método de elementos finitos (FEM) para modelar la propagación del agua en el suelo, permitiendo simular la irrigación. Se evalúo múltiples variables como la temperatura ambiente, la humedad del suelo y la hora del día, las cuales controlan la intensidad del riego a través de un sistema de control lógico difuso (FLC). Mediante un modelo 2D, lograron simular y comparar distintos sistemas de riego, facilitando la elección del más eficiente para suministrar agua a las raíces de las plantas con el menor consumo posible. El uso de un controlador de lógica difusa, basado en variables como temperatura, humedad del suelo y hora del día, permite determinar el tiempo y la cantidad óptima de riego. Las pruebas incluyeron la ubicación de sensores de humedad del suelo a diferentes profundidades y su impacto en el consumo de agua. Además, se realizaron pruebas comparativas entre sistemas de riego convencionales y aquellos controlados por lógica difusa bajo diversas condiciones y distribuciones de variables. En estas pruebas se observó que el sistema controlado por lógica difusa utilizó un 13\% menos de agua en comparación con el sistema convencional. Sin embargo, en una simulación con distribuciones diferentes de variables pero con las mismas reglas, se notó un aumento considerable en el consumo de agua por parte del sistema controlado por lógica difusa. Sin embargo, sería valioso proporcionar resultados más detallados obtenidos de las comparaciones entre diferentes sistemas de riego controlados por lógica difusa. 

\bigbreak
En \cite{ramos_galindo_diseno_2023} se basó en el desarrollo de una aplicación móvil utilizando el framework Flutter para el monitoreo de humedad del suelo. Este sistema se complementa con un conjunto de sensores controlados por un microcontrolador ESP32 y hace uso de Firebase como base de datos en tiempo real. La arquitectura utilizada implica el uso de Flutter como framework principal para la creación de la aplicación móvil, permitiendo el desarrollo de interfaces intuitivas y la visualización de datos en tiempo real provenientes de sensores. El microcontrolador ESP32 se elige por su capacidad integral para manejar las necesidades del proyecto, incluyendo la adquisición de datos analógicos de los sensores de humedad. En cuanto a la base de datos y autenticación, Firebase es seleccionado por su capacidad de mantener operativa una base de datos en tiempo real las 24/7. Esto permitió almacenar y gestionar los datos obtenidos por los sensores de manera eficiente y accesible. El proceso inicio con la adquisición de datos a través de sensores de humedad conectados al microcontrolador ESP32, que capturan información en tiempo real sobre las condiciones del suelo. Estos datos se transmiten a la base de datos en Firebase para su procesamiento y almacenamiento. La aplicación móvil desarrollada en Flutter permite a los usuarios visualizar esta información de manera intuitiva y tomar decisiones basadas en las necesidades específicas de cada cultivo. La arquitectura general del sistema comprende el uso de sensores conectados al microcontrolador ESP32 para la adquisición de datos, Firebase para el almacenamiento en la nube, y Flutter para la interfaz de usuario y visualización de datos en tiempo real. Este sistema busca proporcionar una solución integral para el monitoreo de humedad del suelo, permitiendo a agricultores, horticultores y entusiastas del cultivo acceder fácilmente a información valiosa para mejorar el cuidado de sus cultivos. Sin embargo, podría haber brindado una mayor exploración en los detalles técnicos de la integración de los sensores con el microcontrolador ESP32 y la transmisión de datos a Firebase.

\bigbreak
Tras revisar diversos antecedentes, se destaca cómo la lógica difusa ha tenido un gran impacto al interpretar datos complejos, como la humedad, la temperatura y los niveles de humedad. Esta técnica resulta especialmente útil para representar estados en motores y sistemas difusos, sobre todo en situaciones donde los datos de entrada y salida no son precisos. Además, se ha observado un avance considerable en la precisión y diversidad de sensores, desde aquellos diseñados para medir la humedad del suelo, como el Módulo HL 69, SM3002B y el sensor YL-69, hasta sensores específicos para la lluvia, como el sensor de lluvia y el sensor de gota de lluvia YL-83.
\bigbreak
A pesar de la relevancia que muestran los estudios revisados sobre la aplicación de la lógica difusa en sistemas y sensores, su enfoque no se centra en su implementación directa en el desarrollo de aplicaciones móviles específicamente para el control y monitoreo de riego.
\bigbreak
Es relevante destacar que a pesar de la importancia que tiene el riego de agua en viveros para el cuidado de las plantas, los estudios revisados no abordan específicamente esta práctica en ese entorno. Esta falta de información resalta la necesidad de profundizar en la implementación de sistemas de riego en viveros.
